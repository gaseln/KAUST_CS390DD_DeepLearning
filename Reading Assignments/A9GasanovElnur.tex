\documentclass{article}
\usepackage{amsthm, amsmath, amssymb}
\title{Assignment 9 : Universality theorem and PointNet++}
\author{Elnur Gasanov}
\date{}
\begin{document}
\maketitle
\begin{enumerate}
	\item The author has provided a clear proof outline of universality theorem. But the given "proof" arises some questions. For example, large $w$ generates a sigmoid step-function, which is in the core of proof outline. Does it mean that the weights of the neural network  in the universality theorem always have high magnitude? If so, then the network overfits, what we usually try to avoid in real life. I think, the proof is good, no further improvements are needed. It is relatively easy to catch the main idea. 
	\item PointNet++ in contrast to PointNet learns local features of point clouds gradually increasing the size of analyzed neighbourhood. Possible justification may lie in the fact that point clouds may have different density across the space, so that they have different local features across the space. Results show that hierarchical PointNet significantly outperforms vanilla one on classification and segmentation tasks. 
	
\end{enumerate}
\end{document}