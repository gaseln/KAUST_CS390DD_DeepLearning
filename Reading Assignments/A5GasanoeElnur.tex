\documentclass{article}
\usepackage{graphicx}
\usepackage{amssymb, amsmath, amsthm}

\author{Elnur Gasanov}
\title{Assignment 5 : ImageNet / GAN}
\date{}

\begin{document}
\maketitle

\section{Task}

\begin{enumerate}
	\item Read the paper: Olga Russakovsky et al., ImageNet Large Scale Visual Recognition Challenge
	\item Answer Question: \begin{enumerate} \item Describe the ImageNet dataset with some metrics: How many images does it contain? \item  What is the resolution of images in the dataset? \item How to split between test, training, and validation data? \item What was the impact of the ImageNet dataset? \item How many categories of images does the dataset contain? \item What is ILSVRC? \item What are the challenge tasks? \item Who provided the ground truth labels in the dataset? \end{enumerate} 
	\item Read the paper: Goodfellow et al., Generative Adversarial Networks. Note: The second paper is a bit more challenging to read
	\item Answer Question: \begin{enumerate} \item What is KL( ... )? \item What should the training converge to in the optimal case? \item What is mode collapse (discussed in Section 6 of the paper) \end{enumerate} 
\end{enumerate}

\section{Solutions}

\begin{enumerate}
	\item \begin{enumerate}
	\item Dataset had contained 1,461,406 images by 2010.
	\item The images vary in resolution.
	\item Approximately 1.2 million training images, 50 thousand validation and 100 thousand testing images.
	\item Facilitated the improvement of computer vision algorithms (human accuracy on image classification task is already beaten)
	\item 1000 categories
	\item ImageNet Large Scale Visual Recognition Challenge - competition based on ImageNet dataset
	\item Image classification (which classes are presented in an image), single-object localization (algorithms produce a list of object categories present in the im- age, along with an axis-aligned bounding box indicating the position and scale of one instance of each object category), object detection (the same as the previous task, but show all objects of a class on an image)
	\item Humans through Amazon Mechanical Turk platform 
\end{enumerate}

\item \begin{enumerate}
	\item Kullback Leibler divergence is a measure of how one probability distribution is different from a second, reference probability distribution (from WikiPedia).
	\item $p_g$ should converge to $p_{data}$.
	\item An issue which arises when generator trained "more frequently": in this case generator learns to generate only one mode (only one class) losing the diversity of data.
\end{enumerate}


\end{enumerate}

\end{document}