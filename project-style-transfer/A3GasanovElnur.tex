\documentclass{article}
\usepackage{amssymb, amsmath, amsthm}
\title{Assignment 3: Style Transfer Texture Synthesis}
\author{Elnur Gasanov}
\date{}
\parindent=0.0mm

\begin{document}
\maketitle

{\bfseries Task.}
\begin{enumerate}
	\item Read the paper: Gatys, Ecker, and Bethge, “Image Style Transfer Using Convolutional Neural Networks”, CVPR 2016
	\item Read the paper: Sendik, and Cohen-Or, “Deep Correlations for Texture synthesis”, Siggraph 2017
	\item Pick one of the papers and present a good idea to extend the method. An extension could be seen in the sense of proposing a research project. It can be an idea to make the method more general, apply to different types of data, improve the quality, accelerate the method etc.
\end{enumerate}
	
{\bfseries Idea.}

Let us pick style transfer project. One can try to transfer coarse features (styles) of one style image and fine features of another style image to an arbitrary content image. To do this, we will construct loss function consisting of three terms:

$$
\mathcal{L} = \alpha \mathcal{L}_{coarse} + \beta \mathcal{L}_{fine} + \gamma \mathcal{L}_{content}.
$$

The third term is the same as one in the paper. Remaining loss terms measure Frobenius norm of difference between Gram matrices of resulting and style images. To transfer coarse styles, Gram matrix of resulting image is made similar to Gram matrix of feature maps within the first layer (e.g. conv1\_1) of the network (e.g.~VGG16). The same similarity is chased between feature maps of further layers in order to fetch fine features (e.g. conv3\_2). Hyperparameters $\alpha, \beta, \gamma$ regulate how much of resemblance one wants to get. 
\end{document}